\documentclass[aspectratio=169]{beamer}
\usepackage[portuguese]{babel}
\usepackage{xcolor} % Para colorir o código
\usepackage{booktabs} % Para a formatação da tabela
\usepackage{array}    % Para a personalização de colunas
\usepackage{pgfplots} % Pacote para gráficos
\usepackage{amsmath}

\usetheme{DarkConsole}

\title{Precificação de ativos de renda variável}
\subtitle{Uma Abordagem Baseada em Dados Históricos}
\author{\\Cauã Wendel Sousa da Silva \\ Hertz Rafael Queiroz da Cruz}

\begin{document}

% Slide de título
\begin{frame}
  \maketitle
\end{frame}


% Sumário
\begin{frame}{Sumário}
  \tableofcontents
\end{frame}

% Introdução
\section{Introdução}
\begin{frame}{Introdução}

  \begin{center}
    {\Large Como identificar quais os melhores ativos a serem investidos através de dados estatísticos?}

    
  \end{center}

  \vspace{0.5cm}
Neste trabalho, buscamos responder a essa questão utilizando métodos estatísticos que permitem analisar tendências e históricos de ações, a fim de auxiliar o investidor na tomada de decisões.


  
\end{frame}

% Revisão da Literatura
\section{Revisão da Literatura}
\begin{frame}{Revisão da Literatura}
  \begin{itemize}
    \item Ativos de Renda Variável: Ativos de renda variável são aqueles cujo retorno não é fixo ou garantido, variando conforme as condições de mercado, como ações e alguns fundos de investimentos.

    \item Oferecem maior potencial de retorno, porém com maior risco.

    \item Para proteção do investidor, a 'Modern Portfolio Theory'
    de Markowitz (1952) surge como uma estrutura de investimento para a seleção e
    construção de carteiras de investimentos com base na maximização dos retornos esperados da carteira e a minimização simultânea do risco de investimento. \cite{mangram2013simplified}

  \end{itemize}
\end{frame}

\begin{frame}{Revisão da Literatura}
  \begin{itemize}
    \item Índice de Sharpe: Avalia a relação entre risco e retorno, levando em consideração uma taxa livre de risco. Esse índice fornece ao investidor a informação necessária para determinar se os riscos de possíveis perdas são compensados pelo retorno esperado. \cite{de2015fundos}

    \item Bandas de Bollinger: Serve para identificar quando o investidor deve comprar ou vender suas ações a fim de maximizar seus lucros, comprando enquanto está desvalorizado e vendendo supervalorizado. \cite{yan2023enhanced}

    \item Desvio Padrão: Caracteriza a distância típica de uma observação do centro da distribuição; em outras palavras, reflete a dispersão das observações individuais da amostra em torno da média da amostra. \cite{curran1998fundamental}

    \item Média Móvel: Resume os padrões gerais de mudança ao longo de um período passado e é frequentemente utilizada para prever a tendência futura de séries temporais. \cite{su2022self}

  \end{itemize}
\end{frame}

% Objetivos
\section{Objetivos}
\begin{frame}{Objetivos}
  \textbf{Geral:}
  Obter insights que possam ajudar o investidor a identificar em qual ativo pode ser mais rentável investir, com base em dados históricos e estatísticas.

  \vspace{0.5cm}
  \textbf{Específicos:}
  \begin{itemize}
    \item Utilizar o Índice de Sharpe para comparar o retorno ajustado ao risco entre diferentes investimentos.
    \item Calcular a Média Móvel para identificar tendências de preço ao longo do tempo.
    \item Aplicar as Bandas de Bollinger para detectar condições de valorização ou desvalorização nos ativos.
  \end{itemize}
\end{frame}

% Metodologia
\section{Metodologia}
\begin{frame}{Metodologia}
  \textbf{Ferramentas e Procedimentos Utilizados:}
  \begin{itemize}
    \item Coleta de dados financeiros via biblioteca do \texttt{yfinance}.
    \item Transformação e limpeza dos dados do DataFrame utilizando \texttt{pandas}, incluindo padronização das colunas.
    \item Consideração do retorno sobre o risco através do Índice de Sharpe.
    \item Utilização da Média Móvel para identificar a tendência do valor de um ativo.
    \item Aplicação das bandas de Bollinger para avaliação de valorização ou desvalorização dos ativos.
  \end{itemize}
\end{frame}

% Resultado
\section{Resultado}

\begin{frame}{Resultados}

\begin{center}
\[
\text{Índice de Sharpe} = \frac{\text{Retorno Acumulado} - \text{Taxa Livre de Risco}}{\text{Desvio Padrão}}
\]
\end{center}

\vspace{0.5em}

\begin{center}
Taxa livre de risco: SELIC (13,75\%)
\end{center}


\begin{table}[h]
    \centering
    \large
    \begin{tabular}{|c|c|} % Define duas colunas com bordas verticais
        \hline % Linha horizontal
        Ativo  & IS \\ % Cabeçalho da tabela
        \hline % Linha horizontal
        Nvidia (NVDA)   & 16,58   \\ % Linha 1
        \hline % Linha horizontal
        Apple (AAPL)   & 3,58   \\ % Linha 2
        \hline % Linha horizontal
        Banco do Brasil (BBAS3)  & 3,31   \\ % Linha 3
        \hline % Linha horizontal
        Petrobrás (PETR4)  & 1,69   \\ % Linha 4
        \hline % Linha horizontal
        Tesla (TSLA)   & -0,5  \\ % Linha 5
        \hline % Linha horizontal
    \end{tabular}
    \caption{Ranking Índice de Sharpe (27/09/2023 - 26/09/2024)}
    \label{tab:ranking_sharpe}
\end{table}
\end{frame}

\begin{frame}{Identificando Tendências de Preço}
\begin{center}
    Analisando ações da Nvidia    
\end{center}

\begin{center}
    Período: 28/08/2024 - 26/09/2024 \\
    Valor Médio: 115
\end{center}
    
\begin{figure}
    \centering
    \begin{tikzpicture}
        \begin{axis}[
            title={Gráfico do Valor de Fechamento do Ativo},
            xlabel={Dias},
            ylabel={Valor},
            width=15cm,
            height=5cm,
            grid=major,
            minor tick num=1,
            xtick={1, ..., 21},
            xmin=1,
            xmax=21,
            smooth,
            extra y ticks={115},
            extra y tick style={tick label style={color=yellow}} % Suaviza a linha % Estilo do tick amarelo
        ]
            \addplot[color=cyan, line width=2pt, mark=none] table [col sep=comma, x=index, y=close]{NVDA.csv};

            \addplot[color=yellow, line width=2pt, dashed] 
                coordinates {(1,115) (21,115)};
        \end{axis}
    \end{tikzpicture}
\end{figure}
\end{frame}

\begin{frame}{Bandas de Bollinger}

\[
\text{Bandas de Bollinger} = \text{Média Móvel} \pm k \cdot \text{Desvio Padrão}
\]

\begin{center}
    Analisando ações da Tesla  
\end{center}

\begin{center}
    Período: 28/09/2021 - 26/09/2024 \\
\end{center}
    
\begin{table}[h]
    \centering
    \large
    \begin{tabular}{|c|c|c|} % Define duas colunas com bordas verticais
        \hline % Linha horizontal
        Data / Período  & Valor & Status \\ % Cabeçalho da tabela
        \hline % Linha horizontal
        04/11/2021  & 409.97 & Super Valorizado   \\ % Linha 1
        \hline % Linha horizontal
        28/09/2021 - 26/09/2024 & 238.26  & Valor médio  \\ % Linha 2
        \hline % Linha horizontal
        03/01/2023  & 108.10 & Desvalorizado   \\ % Linha 3
        \hline % Linha horizontal
    \end{tabular}
    \caption{Valores das ações da Tesla com base nas Bandas de Bollinger.}
    \title{Bordas de Bollinger}
    \label{tab:bollinger_table}
\end{table}

\begin{center}
    
 O desvio padrão é de 59.72, média movel é de 238.26 e usando k = 2. Qualquer valor acima de 357,7 é considerado super valorizado e abaixo de 118,82 é considerado desvalorizado.

\end{center}

\end{frame}
    
% Conclusão
\section{Conclusão}
\begin{frame}{Contribuição do Trabalho}
  Este trabalho apresenta uma abordagem estatística que permite ao investidor extrair insights valiosos com base em dados históricos e tendências. Utilizando métodos como o Índice de Sharpe, a Média Móvel e as Bandas de Bollinger é possível que o usuário tome decisões mais informadas e aumente suas chances de identificar ativos com maior potencial de rentabilidade.
\end{frame}

\bibliographystyle{plain}
\bibliography{ref}
\end{document}